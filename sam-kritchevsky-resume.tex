%-------------------------
% Rezume, a latex resume template for developers
% Author : Nanu Panchamurthy
% Based off of: https://github.com/sb2nov/resume
% License : MIT

% Hope this resume template helps you land an awesome job. If you found this helpful, please consider starring the github repo here, .
%-------------------------



%------------PACKAGES----------------
\documentclass[a4paper,10.5pt]{article}

\usepackage{verbatim} % reimplements the "verbatim" and "verbatim*" environments

\usepackage{titlesec} % provides an interface to sectioning commands i.e. custom elements

\usepackage[dvipsnames]{xcolor} % provides both foreground and background color management

\usepackage{enumitem} % provides control over enumerate, itemize and description

\usepackage{fancyhdr} % provides extensive facilities for constructing headers, footers and also controlling their use

\usepackage{tabularx} % defines an environment tabularx, extension of "tabular" with an extra designator x, paragraph like column whose width automatically expands to fill the width of the environment

\usepackage{latexsym} % provides mathematical symbols

\usepackage{marvosym} % provides martin vogel's symbol font which contains various symbols

\usepackage[empty]{fullpage} % sets margins to one inch and removes headers, footers etc..

\usepackage[hidelinks]{hyperref} % removes color and shadow of hyperlinks

\usepackage[normalem]{ulem} % provides "\ul" (uline) command which will break at line breaks

\usepackage[english]{babel} % provides culturally determined typographical rules for wide range of languages

%-----------------------------------------

% \input glyphtounicode % converts glyph names to unicode
% \pdfgentounicode=1 % ensures pdfs generated are ats readable

%----------FONT OPTIONS-------------------
\usepackage[default]{sourcesanspro} % uses the font source sans pro
\urlstyle{same} % changes url font from default urlfont to font being used by the document
%-----------------------------------------


%----------MARGIN OPTIONS-----------------
\pagestyle{fancy} % set page style to one configured by fancyhdr
\fancyhf{} % clear all header and footer fields

\renewcommand{\headrulewidth}{0in} % sets thickness of linerule under header to zero
\renewcommand{\footrulewidth}{0in} % sets thickness of linerule over footer to zero

\setlength{\tabcolsep}{0in} % sets thickness of column separator in tables to zero

% origin of the document is one inch from the top and from and the left
% oddsidemargin and evensidemargin both refer to the left margin
% right margin is indirectly set using oddsidemargin
\addtolength{\oddsidemargin}{-0.5in}
\addtolength{\topmargin}{-0.5in}

\addtolength{\textwidth}{1.0in} % sets width of text area in the page to one inch
\addtolength{\textheight}{1.0in} % sets height of text area in the page to one inch

\raggedbottom{} % makes all pages the height of current page, no extra vertical space added
\raggedright{} % makes all pages the width of current page, no extra horizontal space added
%------------------------------------------


%--------SECTIONING COMMANDS---------
% \titleformat{<command>}
%   [<shape>]{<format>}{<label>}{<sep>}
%     {<before-code>}[<after-code>]

% command is the sectioning command to be redefined
% shape is the style of the font; scshape stands for small caps style
% format is the format to be applied to whole title- label and text; absent here
% label defines the label
% sep is the horizontal separation between label and title body
% before-code is the code to be executed before
% after-code is the code to be executed after

\titleformat{\section}
  {\scshape\large}{}
    {0em}{\color{BrickRed}}[\color{black}\titlerule\vspace{0pt}]
%-------------------------------------


%--------REDEFINITIONS----------------
% redefines the style of the bullet point
\renewcommand\labelitemii{$\vcenter{\hbox{\tiny$\bullet$}}$}

% redefines the underline depth to 2pt
\renewcommand{\ULdepth}{2pt}
%-------------------------------------


%--------CUSTOM COMMANDS--------------
%\vspace{} defines a vertical space of given size, modifying this in custom commands can help stretch or shrink resume to remove or add content

% resumeItem renders a bullet point
\newcommand{\resumeItem}[1]{
  \item\small{#1}
}

% commands to start and end itemization of resumeItem, rightmargin set to 0.11in to avoid the overflow of resumetItem beyond whatever resumeItemHeading is being used
\newcommand{\resumeItemListStart}{\begin{itemize}[rightmargin=0.11in]}
\newcommand{\resumeItemListEnd}{\end{itemize}}

% resumeSectionType renders a bolded type to be used under a section, used as skill type here, middle element is used to keep ":"s in the same vertical line
\newcommand{\resumeSectionType}[3]{
  \item\begin{tabular*}{0.96\textwidth}[t]{
    p{0.15\linewidth}p{0.02\linewidth}p{0.81\linewidth}
  }
    \textbf{#1} & #2 & #3
  \end{tabular*}\vspace{-2pt}
}

\newcommand{\resumeSingleHeading}[1]{
  \item\small{
    \begin{tabular*}{0.96\textwidth}[t]{
      l@{\extracolsep{\fill}}c@{\extracolsep{\fill}}r
    }
      \textbf{#1}
    \end{tabular*}
  }
}

% resumeTrioHeading renders three elements in three columns with second element being italicized and first element bolded, can be used for projects with three elements
\newcommand{\resumeTrioHeading}[3]{
  \item\small{
    \begin{tabular*}{0.96\textwidth}[t]{
      l@{\extracolsep{\fill}}c@{\extracolsep{\fill}}r
    }
      \textbf{#1} & \textit{#2} & #3
    \end{tabular*}
  }
}

% resumeQuadHeading renders four elements in a two columns with the second row being italicized and first element of first row bolded, can be used for experience and projects with four elements
\newcommand{\resumeQuadHeading}[4]{
  \item
  \begin{tabular*}{0.96\textwidth}[t]{l@{\extracolsep{\fill}}r}
    \textbf{#1} & #2 \\
    \textit{\small#3} & {\small #4} \\
  \end{tabular*}
}

% resumeQuadHeadingChild renders the second row of resumeQuadHeading, can be used for experience if different roles in the same company need to added
\newcommand{\resumeQuadHeadingChild}[2]{
  \item
  \begin{tabular*}{0.96\textwidth}[t]{l@{\extracolsep{\fill}}r}
    \textit{\small#1} & {\small#2} \\
  \end{tabular*}
}

% commands to start and end itemization of resumeQuadHeading, lefmargin for left indent of 0.15in for resumeItems
\newcommand{\resumeHeadingListStart}{
  \begin{itemize}[leftmargin=0.15in, label={}]
}
\newcommand{\resumeHeadingListEnd}{\end{itemize}}
%-------------------------------------------


%__________________RESUME____________________
% You can rearrange sections in any order you may prefer
\begin{document}

%-----------CONTACT DETAILS------------------
% Make sure all the details are correct, you can add more links in the first row of second column if needed

\begin{tabular*}{\textwidth}{l@{\extracolsep{\fill}}r}
  \Huge Sam Kritchevsky \vspace{2pt} & % row = 1, col = 1
  \\% row = 1, col = 2
  \href{https://github.com/skritch}{\uline{GitHub}} $|$
  \href{https://samkrit.ch/}{\uline{Website}} $|$
  \href{https://www.linkedin.com/in/sam-kritchevsky-0b4501122/}{\uline{LinkedIn}} $|$
  \href{mailto:sam.kritch@gmail.com}{\uline{sam.kritch@gmail.com}} $|$
  336-414-8694 $|$
  NYC, NC, or Remote $|$ Full-time or Contract\\ 
\end{tabular*}
%--------------------------------------------


%-----------SUMMARY--------------------------
% Keep this short, simple and straight to point

\section{Professional Summary}
\small{\setlength{\parindent}{1em}\indent
  I am a software engineer and data scientist with 5 years of professional experience. I previously worked at SeatGeek, where I was the tech lead for a six-member Data Platform team after serving in various other roles across the data organization during my tenure. 
  I am most at home in the domain of data engineering, but I also have plenty of experience with ML-eng, Data Science, Analytics, backend, and full-stack engineering—I often describe myself as a "data generalist".
  
  Before my tech career I was enrolled in a Ph.D. program in physics, and I still maintain a hobbyist interest in physics, math and math education, and occasionally still write essays on these topics, which can be seen on my personal site.

  Please note that I am currently re-entering the tech industry after a few years away. In the meantime I have been working on personal projects, writing on my \href{http://sketchingtowards.substack.com/}{\uline{Substack}}, traveling, taking piano lessons, and working as a private math and physics tutor.
}
%--------------------------------------------


%--------------SKILLS------------------------
% Add or remove resumeSectionTypes according to your needs

\section{Technical Skills}
  \resumeHeadingListStart{}
    \resumeSectionType{Languages}{:}{Python, SQL. Some Typescript, Javascript, Scala.}
    \resumeSectionType{Tech}{:}{PostgreSQL, DBT, Spark, Flink, RabbitMQ, Elasticsearch, Redis, PostGIS, Pandas, Numpy, Tensorflow, Docker, Luigi, Dagster, Akka, React, Svelte, Docker, Nomad}
    \resumeSectionType{Cloud Tech}{:}{Various AWS, incl. Redshift, Redshift Spectrum, Kinesis, Kinesis Firehose, S3, Lambda, EMR, Glue, IAM, RDS.}
    \resumeSectionType{Roles}{:}{Data Engineering, Data Platform, Backend Engineering, ML Engineering, Data Science, Analytics}
  \resumeHeadingListEnd{}
%--------------------------------------------


%-----------EXPERIENCE-----------------------
% Distill all your talking points to small bullet points which follow the pattern "challenge-action-result" for maximum efficiency. Try to quantify (use numbers) your points whenver possible, highlist words of importance

\section{Experience}
\resumeHeadingListStart{}
  \resumeQuadHeading{SeatGeek}{2016-2021}
  {Sr. Software Engineer, Data Platform}{2020-21}
  Tech lead of the Data Platform team. Our projects included:
  \resumeItemListStart{}
    \resumeItem{A rewrite of our clickstream ingestion service, massively improving scalability and fault tolerance, particularly during spiky event on-sales, and enabling fanout to many downstream applications. Tech: Flink, Spark}
    \resumeItem{Established a standardized notebook stack for use by our data science teams, which allowed DS analysis work to be shared reproducibly across the team and to be incorporated into production data pipelines. Tech: Jupyter, Papermill}
    \resumeItem{Rolled out a centralized system data documentation, lineage, and discovery. Tech: \href{https://github.com/amundsen-io/amundsen}{\uline{Amundsen}}, Airflow, AWS Neptune.}
    \resumeItem{Owned and maintained the streaming data pipelines for ticket inventory, search, clickstream, and experimentation data. Tech: Spark, Akka Streaming, Python, AWS Kinesis.}
    \resumeItem{Owned and maintained the core analytics stack, which backed our BI layer and served data needs for many departments. Tech: \href{https://github.com/seatgeek/druzhba}{\uline{Druzhba}}, DBT, Luigi, Redshift, Looker.}
  \resumeItemListEnd{}

  \resumeQuadHeadingChild{Senior Data Scientist}{2019-20}
  \resumeQuadHeadingChild{Data Scientist}{2017-19}

  I specialized on user preference signals, our live event catalog, and push marketing. Projects included:
  \resumeItemListStart{}
  \resumeItem{Recommendation algorithms for weekly newsletter and various cart abandonment and price-drop notifications, tripling the revenue attributed to email notifications. Tech: Spark on AWS EMR, SQL.}
  \resumeItem{Event and performer popularity models, feeding into event recommendations and searcc. Tech: Tensorflow, Keras, Numpy.}
  \resumeItem{Rewrite of entity-linking algorithm, which deduplicated users and device for marketing attribution and use in funnel KPIs. Tech: SQL, Luigi.}
  \resumeItem{Ongoing involvement in the design and measurement of metrics for recommendations, search, and push marketing.}
  \resumeItemListEnd{}

  \resumeQuadHeadingChild{Software Engineer, Discovery}{2016-17}
  \resumeItemListStart{}
  \resumeItem{Backend engineering and data analysis for frontpage recommendations, search, social features, and push marketing.}
  \resumeItem{I built a simple reverse-ETL system to push datasets from our Data Warehouse into production PostgreSQL and MySQL databases.}
  \resumeItemListEnd{}

  \resumeSingleHeading{Other}
  \resumeQuadHeadingChild{Private Math Tutor, Club Z Tutoring}{2024-2025}
  \resumeQuadHeadingChild{Volunteer Data Analyst, Greater Harlem Coalition}{2022}
  Contributed data analysis, writing, editing, and website administration for a community advocacy group, using a data pipeline built with DBT, Python, Dagster, and PostGIS (code \href{https://github.com/skritch/ghc-analysis}{\uline{here}}.)
  \resumeQuadHeadingChild{Physics Graduate Student and Teaching Assistant}{2014-2016}
  My research interests involved statistical mechanics and dynamical systems.
\resumeHeadingListEnd{}
%---------------------------------------------




%-----------EDUCATION-------------------------

\section{Education}
  \resumeHeadingListStart{}
    \resumeQuadHeading{University of Wisconsin}{Madison, WI}
    {Physics Ph.D. Student}{2013-2015}
    \resumeQuadHeading{University of North Carolina}{Chapel Hill, NC}
    {B.S. Physics, Math Minor}{2012}
  \resumeHeadingListEnd{}
%---------------------------------------------


\setlength{\footskip}{4.08003pt}
\end{document}