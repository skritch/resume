%-------------------------
% Rezume, a latex resume template for developers
% Author : Nanu Panchamurthy
% Based off of: https://github.com/sb2nov/resume
% License : MIT

% Hope this resume template helps you land an awesome job. If you found this helpful, please consider starring the github repo here, .
%-------------------------



%------------PACKAGES----------------
\documentclass[a4paper,11pt]{article}

\usepackage{verbatim} % reimplements the "verbatim" and "verbatim*" environments

\usepackage{titlesec} % provides an interface to sectioning commands i.e. custom elements

\usepackage[dvipsnames]{xcolor} % provides both foreground and background color management

\usepackage{enumitem} % provides control over enumerate, itemize and description

\usepackage{fancyhdr} % provides extensive facilities for constructing headers, footers and also controlling their use

\usepackage{tabularx} % defines an environment tabularx, extension of "tabular" with an extra designator x, paragraph like column whose width automatically expands to fill the width of the environment

\usepackage{latexsym} % provides mathematical symbols

\usepackage{marvosym} % provides martin vogel's symbol font which contains various symbols

\usepackage[empty]{fullpage} % sets margins to one inch and removes headers, footers etc..

\usepackage[hidelinks]{hyperref} % removes color and shadow of hyperlinks

\usepackage[normalem]{ulem} % provides "\ul" (uline) command which will break at line breaks

\usepackage[english]{babel} % provides culturally determined typographical rules for wide range of languages

%-----------------------------------------

% \input glyphtounicode % converts glyph names to unicode
% \pdfgentounicode=1 % ensures pdfs generated are ats readable

%----------FONT OPTIONS-------------------
\usepackage[default]{sourcesanspro} % uses the font source sans pro
\urlstyle{same} % changes url font from default urlfont to font being used by the document
%-----------------------------------------


%----------MARGIN OPTIONS-----------------
\pagestyle{fancy} % set page style to one configured by fancyhdr
\fancyhf{} % clear all header and footer fields

\renewcommand{\headrulewidth}{0in} % sets thickness of linerule under header to zero
\renewcommand{\footrulewidth}{0in} % sets thickness of linerule over footer to zero

\setlength{\tabcolsep}{0in} % sets thickness of column separator in tables to zero

% origin of the document is one inch from the top and from and the left
% oddsidemargin and evensidemargin both refer to the left margin
% right margin is indirectly set using oddsidemargin
\addtolength{\oddsidemargin}{-0.5in}
\addtolength{\topmargin}{-0.5in}

\addtolength{\textwidth}{1.0in} % sets width of text area in the page to one inch
\addtolength{\textheight}{1.0in} % sets height of text area in the page to one inch

\raggedbottom{} % makes all pages the height of current page, no extra vertical space added
\raggedright{} % makes all pages the width of current page, no extra horizontal space added
%------------------------------------------


%--------SECTIONING COMMANDS---------
% \titleformat{<command>}
%   [<shape>]{<format>}{<label>}{<sep>}
%     {<before-code>}[<after-code>]

% command is the sectioning command to be redefined
% shape is the style of the font; scshape stands for small caps style
% format is the format to be applied to whole title- label and text; absent here
% label defines the label
% sep is the horizontal separation between label and title body
% before-code is the code to be executed before
% after-code is the code to be executed after

\titleformat{\section}
  {\scshape\large}{}
    {0em}{\color{BrickRed}}[\color{black}\titlerule\vspace{0pt}]
%-------------------------------------


%--------REDEFINITIONS----------------
% redefines the style of the bullet point
\renewcommand\labelitemii{$\vcenter{\hbox{\tiny$\bullet$}}$}

% redefines the underline depth to 2pt
\renewcommand{\ULdepth}{2pt}
%-------------------------------------


%--------CUSTOM COMMANDS--------------
%\vspace{} defines a vertical space of given size, modifying this in custom commands can help stretch or shrink resume to remove or add content

% resumeItem renders a bullet point
\newcommand{\resumeItem}[1]{
  \item\small{#1}
}

% commands to start and end itemization of resumeItem, rightmargin set to 0.11in to avoid the overflow of resumetItem beyond whatever resumeItemHeading is being used
\newcommand{\resumeItemListStart}{\begin{itemize}[rightmargin=0.11in]}
\newcommand{\resumeItemListEnd}{\end{itemize}}

% resumeSectionType renders a bolded type to be used under a section, used as skill type here, middle element is used to keep ":"s in the same vertical line
\newcommand{\resumeSectionType}[3]{
  \item\begin{tabular*}{0.96\textwidth}[t]{
    p{0.15\linewidth}p{0.02\linewidth}p{0.81\linewidth}
  }
    \textbf{#1} & #2 & #3
  \end{tabular*}\vspace{-2pt}
}

\newcommand{\resumeSingleHeading}[1]{
  \item\small{
    \begin{tabular*}{0.96\textwidth}[t]{
      l@{\extracolsep{\fill}}c@{\extracolsep{\fill}}r
    }
      \textbf{#1}
    \end{tabular*}
  }
}

% resumeTrioHeading renders three elements in three columns with second element being italicized and first element bolded, can be used for projects with three elements
\newcommand{\resumeTrioHeading}[3]{
  \item\small{
    \begin{tabular*}{0.96\textwidth}[t]{
      l@{\extracolsep{\fill}}c@{\extracolsep{\fill}}r
    }
      \textbf{#1} & \textit{#2} & #3
    \end{tabular*}
  }
}

% resumeQuadHeading renders four elements in a two columns with the second row being italicized and first element of first row bolded, can be used for experience and projects with four elements
\newcommand{\resumeQuadHeading}[4]{
  \item
  \begin{tabular*}{0.96\textwidth}[t]{l@{\extracolsep{\fill}}r}
    \textbf{#1} & #2 \\
    \textit{\small#3} & {\small #4} \\
  \end{tabular*}
}

% resumeQuadHeadingChild renders the second row of resumeQuadHeading, can be used for experience if different roles in the same company need to added
\newcommand{\resumeQuadHeadingChild}[2]{
  \item
  \begin{tabular*}{0.96\textwidth}[t]{l@{\extracolsep{\fill}}r}
    \textit{\small#1} & {\small#2} \\
  \end{tabular*}
}

% commands to start and end itemization of resumeQuadHeading, lefmargin for left indent of 0.15in for resumeItems
\newcommand{\resumeHeadingListStart}{
  \begin{itemize}[leftmargin=0.15in, label={}]
}
\newcommand{\resumeHeadingListEnd}{\end{itemize}}
%-------------------------------------------


%__________________RESUME____________________
% You can rearrange sections in any order you may prefer
\begin{document}

%-----------CONTACT DETAILS------------------
% Make sure all the details are correct, you can add more links in the first row of second column if needed

\begin{tabular*}{\textwidth}{l@{\extracolsep{\fill}}r}
  \textbf{\Huge Sam Kritchevsky \vspace{2pt}} & % row = 1, col = 1
  Location: New York, NY \\ % row = 1, col = 2
  \href{https://github.com/skritch}{\uline{GitHub}} $|$ % row = 2, col = 1
  \href{https://www.linkedin.com/in/sam-kritchevsky-0b4501122/}{\uline{LinkedIn}} $|$ % row = 2, col = 1
  \href{mailto:sam.kritch@gmail.com}{\uline{sam.kritch@gmail.com}} $|$ % row = 2, col = 2
  336-414-8694 \\ % row = 2, col = 2
\end{tabular*}
%--------------------------------------------


%-----------SUMMARY--------------------------
% Keep this short, simple and straigth to point

\section{About Me}
\small{
  I'm a backend engineer and "data generalist", with 5 years' experience working across a modern cloud-based data stack in Python, Scala, and SQL, and a strong math background. I'm returning to work after a personal sabbatical, looking for a role at an energetic small-/medium-sized company where I can quickly be productive and strategically impactful, and in opportunities to develop skills in adjacent technical areas such as ML or full-stack engineering.
}
%--------------------------------------------


%--------------SKILLS------------------------
% Add or remove resumeSectionTypes according to your needs

\section{Technical Skills}
  \resumeHeadingListStart{}
    \resumeSectionType{Languages}{:}{Python, Scala, SQL, some Javascript / Typescript}
    \resumeSectionType{Technologies}{:}{PostgreSQL, Redshift, Elasticsearch, Redis, PostGIS, Spark, Flink, RabbitMQ, Akka, DBT, Docker, Nomad, Datadog, Grafana, AWS, EMR, Lambda, Kinesis, S3, Neptune, Jupyter, Luigi, Dagster}
    \resumeSectionType{Roles}{:}{Data Engineering, Data Science, Data Platform, Backend Engineering, Analytics, ML Engineering}
  \resumeHeadingListEnd{}
%--------------------------------------------


%-----------EXPERIENCE-----------------------
% Distill all your talking points to small bullet points which follow the pattern "challenge-action-result" for maximum efficiency. Try to quantify (use numbers) your points whenver possible, highlist words of importance

\section{Experience}
\resumeHeadingListStart{}
  \resumeQuadHeading{Seatgeek}{2016-2021}
  {Sr. Software Engineer, Data Platform Tech Lead}{1.5 Years}

  Tech lead of team of six engineers. Projects included:
  \resumeItemListStart{}
    \resumeItem{Rewrite of our in-house event ingestion service in Flink + Kafka, adding features and scaleability to handle bursty traffic. I advocated for the project, designed the system, and implemented much of the business logic. Functionality included:}
    \resumeItemListStart{}
      \resumeItem{Fanouts to front page recommendations, push notifications, and Redshift, with plans to support an online feature store.}
      \resumeItem{Realtime sessionization, with handling of late-arriving data.}
      \resumeItem{Validation against a repository of JSONSchemas, with alerting in Datadog}
    \resumeItemListEnd{}
    \resumeItem{Rollout of a standardized Jupyter setup based on Docker, Papermill, and NBViewer.}
    \resumeItem{Deployment of Amundsen, an OSS data-discovery platform, with a custom AWS Neptune backend.}
    \resumeItem{Owned and maintained the core data stack of \href{https://github.com/seatgeek/druzhba}{\uline{Druzhba}} (a now-outdated Python ETL tool; I was the primary maintainer and open-sourced the project), DBT, Looker, Luigi, Redshift, and other AWS products.
            }
  \resumeItemListEnd{}

  \resumeQuadHeadingChild{Sr. Data Scientist, Data Scientist II}{2.5 Years}
  \resumeQuadHeadingChild{Data Scientist II}{}

  Specialized in user recommendation signals and our event catalog, working in Spark, SQL, and Python. Projects included:
  \resumeItemListStart{}
  \resumeItem{Algorithms for the weekly newsletter, cart abandonment and price drops notifications, in Spark + SQL. (This work tripled the revenue associated with the weekly newsletter.)}
  \resumeItem{Event and performer popularity models for event recommendations and search, built with Keras}
  \resumeItem{Entity-linking algorithm deduplicating users and device for marketing attribution and funnel KPIs.}
  \resumeItem{Ongoing involvement in design and measurement of metrics for recommendations, search, and CRM marketing.}
  \resumeItemListEnd{}
  
  \resumeQuadHeadingChild{Software Engineer, Discovery}{1 Year}

  Contributed features for search, CRM marketing, and frontpage recommendations.

  \resumeSingleHeading{Other Experience}
  \resumeQuadHeadingChild{Greater Harlem Coalition (occasional volunteer)}{2022}
  Contributed data analysis, writing, editing, and website administration for a community advocacy group. My data pipeline uses DBT, Python, Dagster, and PostGIS (code \href{https://github.com/skritch/ghc-analysis}{\uline{here}}.)
  \resumeQuadHeadingChild{Physics Teaching Assistant}{2014-2016}
\resumeHeadingListEnd{}
%---------------------------------------------




%-----------EDUCATION-------------------------

\section{Education}
  \resumeHeadingListStart{}
    \resumeQuadHeading{University of Wisconsin}{Madison, WI}
    {Coursework towards Physics Ph.D.}{2013-2015}
    \resumeQuadHeading{University of North Carolina}{Chapel Hill, NC}
    {B.S. Physics}{2012}
  \resumeHeadingListEnd{}
%---------------------------------------------


\setlength{\footskip}{4.08003pt}
\end{document}