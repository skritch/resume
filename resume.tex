%-------------------------
% Rezume, a latex resume template for developers
% Author : Nanu Panchamurthy
% Based off of: https://github.com/sb2nov/resume
% License : MIT

% Hope this resume template helps you land an awesome job. If you found this helpful, please consider starring the github repo here, .
%-------------------------



%------------PACKAGES----------------
\documentclass[a4paper,11pt]{article}

\usepackage{verbatim} % reimplements the "verbatim" and "verbatim*" environments

\usepackage{titlesec} % provides an interface to sectioning commands i.e. custom elements

\usepackage{color} % provides both foreground and background color management

\usepackage{enumitem} % provides control over enumerate, itemize and description

\usepackage{fancyhdr} % provides extensive facilities for constructing headers, footers and also controlling their use

\usepackage{tabularx} % defines an environment tabularx, extension of "tabular" with an extra designator x, paragraph like column whose width automatically expands to fill the width of the environment

\usepackage{latexsym} % provides mathematical symbols

\usepackage{marvosym} % provides martin vogel's symbol font which contains various symbols

\usepackage[empty]{fullpage} % sets margins to one inch and removes headers, footers etc..

\usepackage[hidelinks]{hyperref} % removes color and shadow of hyperlinks

\usepackage[normalem]{ulem} % provides "\ul" (uline) command which will break at line breaks

\usepackage[english]{babel} % provides culturally determined typographical rules for wide range of languages
%-----------------------------------------

\input glyphtounicode % converts glyph names to unicode
\pdfgentounicode=1 % ensures pdfs generated are ats readable

%----------FONT OPTIONS-------------------
\usepackage[default]{sourcesanspro} % uses the font source sans pro
\urlstyle{same} % changes url font from default urlfont to font being used by the document
%-----------------------------------------


%----------MARGIN OPTIONS-----------------
\pagestyle{fancy} % set page style to one configured by fancyhdr
\fancyhf{} % clear all header and footer fields

\renewcommand{\headrulewidth}{0in} % sets thickness of linerule under header to zero
\renewcommand{\footrulewidth}{0in} % sets thickness of linerule over footer to zero

\setlength{\tabcolsep}{0in} % sets thickness of column separator in tables to zero

% origin of the document is one inch from the top and from and the left
% oddsidemargin and evensidemargin both refer to the left margin
% right margin is indirectly set using oddsidemargin
\addtolength{\oddsidemargin}{-0.5in}
\addtolength{\topmargin}{-0.5in}

\addtolength{\textwidth}{1.0in} % sets width of text area in the page to one inch
\addtolength{\textheight}{1.0in} % sets height of text area in the page to one inch

\raggedbottom{} % makes all pages the height of current page, no extra vertical space added
\raggedright{} % makes all pages the width of current page, no extra horizontal space added
%------------------------------------------


%--------SECTIONING COMMANDS---------
% \titleformat{<command>}
%   [<shape>]{<format>}{<label>}{<sep>}
%     {<before-code>}[<after-code>]

% command is the sectioning command to be redefined
% shape is the style of the font; scshape stands for small caps style
% format is the format to be applied to whole title- label and text; absent here
% label defines the label
% sep is the horizontal separation between label and title body
% before-code is the code to be executed before
% after-code is the code to be executed after

\titleformat{\section}
  {\scshape\large}{}
    {0em}{\color{blue}}[\color{black}\titlerule\vspace{0pt}]
%-------------------------------------


%--------REDEFINITIONS----------------
% redefines the style of the bullet point
\renewcommand\labelitemii{$\vcenter{\hbox{\tiny$\bullet$}}$}

% redefines the underline depth to 2pt
\renewcommand{\ULdepth}{2pt}
%-------------------------------------


%--------CUSTOM COMMANDS--------------
%\vspace{} defines a vertical space of given size, modifying this in custom commands can help stretch or shrink resume to remove or add content

% resumeItem renders a bullet point
\newcommand{\resumeItem}[1]{
  \item\small{#1}
}

% commands to start and end itemization of resumeItem, rightmargin set to 0.11in to avoid the overflow of resumetItem beyond whatever resumeItemHeading is being used
\newcommand{\resumeItemListStart}{\begin{itemize}[rightmargin=0.11in]}
\newcommand{\resumeItemListEnd}{\end{itemize}}

% resumeSectionType renders a bolded type to be used under a section, used as skill type here, middle element is used to keep ":"s in the same vertical line
\newcommand{\resumeSectionType}[3]{
  \item\begin{tabular*}{0.96\textwidth}[t]{
    p{0.15\linewidth}p{0.02\linewidth}p{0.81\linewidth}
  }
    \textbf{#1} & #2 & #3
  \end{tabular*}\vspace{-2pt}
}

% resumeTrioHeading renders three elements in three columns with second element being italicized and first element bolded, can be used for projects with three elements
\newcommand{\resumeTrioHeading}[3]{
  \item\small{
    \begin{tabular*}{0.96\textwidth}[t]{
      l@{\extracolsep{\fill}}c@{\extracolsep{\fill}}r
    }
      \textbf{#1} & \textit{#2} & #3
    \end{tabular*}
  }
}

% resumeQuadHeading renders four elements in a two columns with the second row being italicized and first element of first row bolded, can be used for experience and projects with four elements
\newcommand{\resumeQuadHeading}[4]{
  \item
  \begin{tabular*}{0.96\textwidth}[t]{l@{\extracolsep{\fill}}r}
    \textbf{#1} & #2 \\
    \textit{\small#3} & {\small #4} \\
  \end{tabular*}
}

% resumeQuadHeadingChild renders the second row of resumeQuadHeading, can be used for experience if different roles in the same company need to added
\newcommand{\resumeQuadHeadingChild}[2]{
  \item
  \begin{tabular*}{0.96\textwidth}[t]{l@{\extracolsep{\fill}}r}
    \textit{\small#1} & {\small#2} \\
  \end{tabular*}
}

% commands to start and end itemization of resumeQuadHeading, lefmargin for left indent of 0.15in for resumeItems
\newcommand{\resumeHeadingListStart}{
  \begin{itemize}[leftmargin=0.15in, label={}]
}
\newcommand{\resumeHeadingListEnd}{\end{itemize}}
%-------------------------------------------


%__________________RESUME____________________
% You can rearrange sections in any order you may prefer
\begin{document}

%-----------CONTACT DETAILS------------------
% Make sure all the details are correct, you can add more links in the first row of second column if needed

\begin{tabular*}{\textwidth}{l@{\extracolsep{\fill}}r}
  \textbf{\Huge Sam Kritchevsky \vspace{2pt}} & % row = 1, col = 1
  Location: New York, NY \\ % row = 1, col = 2
  \href{https://github.com/skritch}{\uline{GitHub}} $|$ % row = 2, col = 1
  \href{https://www.linkedin.com/in/sam-kritchevsky-0b4501122/}{\uline{LinkedIn}} $|$ % row = 2, col = 1
  \href{mailto:sam.kritch@gmail.com}{\uline{sam.kritch@gmail.com}} $|$ % row = 2, col = 2
  336-414-8694 \\ % row = 2, col = 2
\end{tabular*}
%--------------------------------------------


%-----------SUMMARY--------------------------
% Keep this short, simple and straigth to point

\section{About Me}
\small{
  I'm a software engineer and data generalist, with 5 years of experience across a modern cloud-based data stack.
}
%--------------------------------------------


%--------------SKILLS------------------------
% Add or remove resumeSectionTypes according to your needs

\section{Technical Skills}
  \resumeHeadingListStart{}
    \resumeSectionType{Languages}{:}{Python, Scala, SQL, some Javascript / Typescript}
    \resumeSectionType{Technologies}{:}{PostgreSQL, Redshift, DBT, Redis, Spark, Flink, RabbitMQ, Elasticsearch, Docker, Nomad, Datadog, Grafana, AWS, EMR, Lambda, Kinesis, S3, Neptune, Jupyter, Dagster, Luigi, Akka}
    \resumeSectionType{Roles}{:}{Data Science, Data Engineering, Data Platform, Backend, Analytics}
  \resumeHeadingListEnd{}
%--------------------------------------------


%-----------EXPERIENCE-----------------------
% Distill all your talking points to small bullet points which follow the pattern "challenge-action-result" for maximum efficiency. Try to quantify (use numbers) your points whenver possible, highlist words of importance

\section{Experience}
\resumeHeadingListStart{}
  \resumeQuadHeading{Seatgeek}{New York, NY}
  {Sr. Software Engineer, Data Platform, Tech Lead}{2020-2021}

  Tech lead of team of six engineers. Projects I was personally involved with included:
  \resumeItemListStart{}
    \resumeItem{Rewrite of our in-house event ingestion service in Flink + Kafka. Our primary goal was scalability—the previous system was very unstable—but we also added a number of features:}
    \resumeItemListStart{}
    \resumeItem{Fanout to multiple downstreams, including realtime marketing and recommendations, as well as batch loads into Redshift}
    \resumeItem{Realtime sessionization via a stateful Flink job}
    \resumeItem{Realtime validation against JSONSchema}
    \resumeItemListEnd{}
    I advocated for the project, designed the system, and implemented most of the core validation and transformation jobs. Others on the team deployed Flink and Kafka, and built the sessionization job. 
    \resumeItem{Deployment of Amundsen for an OSS data-discovery and documentation platform. We built an AWS Neptune backend and added a number of extensions to ingest lineage metadata from internal services.}
    \resumeItem{Deployed a standard Jupyter Notebook infrastructure based on Docker, Papermill, and NBViewer. This let users include notebooks in production workflows, and supported local development in an identical environment to production.}
    \resumeItem{Extensive improvements to the core data stack: \href{https://github.com/seatgeek/druzhba}{\uline{Druzhba}} (in-house DB-to-DB ETL tool, now open-source)}, Kinesis, DBT, Luigi (heavily modified), and Redshift.
  \resumeItemListEnd{}

  \resumeQuadHeadingChild{Sr. Data Scientist}{2019-2020}
  \resumeQuadHeadingChild{Data Scientist II}{2018}

 I specialized in recommendation signal and user activity data, working in Spark, SQL, and Python.
  \resumeItemListStart{}
    \resumeItem{Implemented email newsletter recommendations algorithm in Spark. This roughly tripled the sales attributed to the weekly newsletter immediately, to about \$700k/week. I also built algorithms for price drop detection and cart-abandonment notifications.}
    \resumeItem{Built event and performer popularity models in Keras, to use as inputs to event recommendations and search.}
    \resumeItem{Implemented customer entity-linking algorithm to deduplicate devices and users in marketing attribution and funnel KPIs.}
    \resumeItem{Ongoing involvement in the design and measurement of recommendation, search, and CRM marketing systems.}
  \resumeItemListEnd{}
  
  \resumeQuadHeadingChild{Software Engineer, Discovery}{Sept 2016-2017}
  \resumeItemListStart{}
  \resumeItem{I built features for search, marketing, and frontpage recommendations using Elasticsearch, Postgres, Redis, RabbitMQ, and Python.}
  \resumeItemListEnd{}

  
  \resumeQuadHeadingChild{Greater Harlem Coalition, Volunteer}{2022-}
  \resumeItemListStart{}
  \resumeItem{Contributed writing and data analysis to advocate for an equitable distribution of social services. Code \href{https://github.com/skritch/ghc-analysis}{\uline{here}}.}
  \resumeItemListEnd{}
\resumeHeadingListEnd{}
%---------------------------------------------




%-----------EDUCATION-------------------------
% Mention your CGPA, if its good, in the first row of second column

\section{Education}
  \resumeHeadingListStart{}
    \resumeQuadHeading{University of Wisconsin}{Madison, WI}
    {Ph.D. Physics (coursework)}{2013-2015}
    \resumeQuadHeading{University of North Carolina}{Chapel Hill, NC}
    {B.S. Physics}{2012}
  \resumeHeadingListEnd{}
%---------------------------------------------


\setlength{\footskip}{4.08003pt}
\end{document}